\documentclass[]{article}
\usepackage{lmodern}
\usepackage{amssymb,amsmath}
\usepackage{ifxetex,ifluatex}
\usepackage{fixltx2e} % provides \textsubscript
\ifnum 0\ifxetex 1\fi\ifluatex 1\fi=0 % if pdftex
  \usepackage[T1]{fontenc}
  \usepackage[utf8]{inputenc}
\else % if luatex or xelatex
  \ifxetex
    \usepackage{mathspec}
  \else
    \usepackage{fontspec}
  \fi
  \defaultfontfeatures{Ligatures=TeX,Scale=MatchLowercase}
\fi
% use upquote if available, for straight quotes in verbatim environments
\IfFileExists{upquote.sty}{\usepackage{upquote}}{}
% use microtype if available
\IfFileExists{microtype.sty}{%
\usepackage{microtype}
\UseMicrotypeSet[protrusion]{basicmath} % disable protrusion for tt fonts
}{}
\usepackage[margin=1in]{geometry}
\usepackage{hyperref}
\hypersetup{unicode=true,
            pdftitle={LAB5 - CP2},
            pdfauthor={bruna barbosa},
            pdfborder={0 0 0},
            breaklinks=true}
\urlstyle{same}  % don't use monospace font for urls
\usepackage{color}
\usepackage{fancyvrb}
\newcommand{\VerbBar}{|}
\newcommand{\VERB}{\Verb[commandchars=\\\{\}]}
\DefineVerbatimEnvironment{Highlighting}{Verbatim}{commandchars=\\\{\}}
% Add ',fontsize=\small' for more characters per line
\usepackage{framed}
\definecolor{shadecolor}{RGB}{248,248,248}
\newenvironment{Shaded}{\begin{snugshade}}{\end{snugshade}}
\newcommand{\KeywordTok}[1]{\textcolor[rgb]{0.13,0.29,0.53}{\textbf{{#1}}}}
\newcommand{\DataTypeTok}[1]{\textcolor[rgb]{0.13,0.29,0.53}{{#1}}}
\newcommand{\DecValTok}[1]{\textcolor[rgb]{0.00,0.00,0.81}{{#1}}}
\newcommand{\BaseNTok}[1]{\textcolor[rgb]{0.00,0.00,0.81}{{#1}}}
\newcommand{\FloatTok}[1]{\textcolor[rgb]{0.00,0.00,0.81}{{#1}}}
\newcommand{\ConstantTok}[1]{\textcolor[rgb]{0.00,0.00,0.00}{{#1}}}
\newcommand{\CharTok}[1]{\textcolor[rgb]{0.31,0.60,0.02}{{#1}}}
\newcommand{\SpecialCharTok}[1]{\textcolor[rgb]{0.00,0.00,0.00}{{#1}}}
\newcommand{\StringTok}[1]{\textcolor[rgb]{0.31,0.60,0.02}{{#1}}}
\newcommand{\VerbatimStringTok}[1]{\textcolor[rgb]{0.31,0.60,0.02}{{#1}}}
\newcommand{\SpecialStringTok}[1]{\textcolor[rgb]{0.31,0.60,0.02}{{#1}}}
\newcommand{\ImportTok}[1]{{#1}}
\newcommand{\CommentTok}[1]{\textcolor[rgb]{0.56,0.35,0.01}{\textit{{#1}}}}
\newcommand{\DocumentationTok}[1]{\textcolor[rgb]{0.56,0.35,0.01}{\textbf{\textit{{#1}}}}}
\newcommand{\AnnotationTok}[1]{\textcolor[rgb]{0.56,0.35,0.01}{\textbf{\textit{{#1}}}}}
\newcommand{\CommentVarTok}[1]{\textcolor[rgb]{0.56,0.35,0.01}{\textbf{\textit{{#1}}}}}
\newcommand{\OtherTok}[1]{\textcolor[rgb]{0.56,0.35,0.01}{{#1}}}
\newcommand{\FunctionTok}[1]{\textcolor[rgb]{0.00,0.00,0.00}{{#1}}}
\newcommand{\VariableTok}[1]{\textcolor[rgb]{0.00,0.00,0.00}{{#1}}}
\newcommand{\ControlFlowTok}[1]{\textcolor[rgb]{0.13,0.29,0.53}{\textbf{{#1}}}}
\newcommand{\OperatorTok}[1]{\textcolor[rgb]{0.81,0.36,0.00}{\textbf{{#1}}}}
\newcommand{\BuiltInTok}[1]{{#1}}
\newcommand{\ExtensionTok}[1]{{#1}}
\newcommand{\PreprocessorTok}[1]{\textcolor[rgb]{0.56,0.35,0.01}{\textit{{#1}}}}
\newcommand{\AttributeTok}[1]{\textcolor[rgb]{0.77,0.63,0.00}{{#1}}}
\newcommand{\RegionMarkerTok}[1]{{#1}}
\newcommand{\InformationTok}[1]{\textcolor[rgb]{0.56,0.35,0.01}{\textbf{\textit{{#1}}}}}
\newcommand{\WarningTok}[1]{\textcolor[rgb]{0.56,0.35,0.01}{\textbf{\textit{{#1}}}}}
\newcommand{\AlertTok}[1]{\textcolor[rgb]{0.94,0.16,0.16}{{#1}}}
\newcommand{\ErrorTok}[1]{\textcolor[rgb]{0.64,0.00,0.00}{\textbf{{#1}}}}
\newcommand{\NormalTok}[1]{{#1}}
\usepackage{graphicx,grffile}
\makeatletter
\def\maxwidth{\ifdim\Gin@nat@width>\linewidth\linewidth\else\Gin@nat@width\fi}
\def\maxheight{\ifdim\Gin@nat@height>\textheight\textheight\else\Gin@nat@height\fi}
\makeatother
% Scale images if necessary, so that they will not overflow the page
% margins by default, and it is still possible to overwrite the defaults
% using explicit options in \includegraphics[width, height, ...]{}
\setkeys{Gin}{width=\maxwidth,height=\maxheight,keepaspectratio}
\IfFileExists{parskip.sty}{%
\usepackage{parskip}
}{% else
\setlength{\parindent}{0pt}
\setlength{\parskip}{6pt plus 2pt minus 1pt}
}
\setlength{\emergencystretch}{3em}  % prevent overfull lines
\providecommand{\tightlist}{%
  \setlength{\itemsep}{0pt}\setlength{\parskip}{0pt}}
\setcounter{secnumdepth}{0}
% Redefines (sub)paragraphs to behave more like sections
\ifx\paragraph\undefined\else
\let\oldparagraph\paragraph
\renewcommand{\paragraph}[1]{\oldparagraph{#1}\mbox{}}
\fi
\ifx\subparagraph\undefined\else
\let\oldsubparagraph\subparagraph
\renewcommand{\subparagraph}[1]{\oldsubparagraph{#1}\mbox{}}
\fi

%%% Use protect on footnotes to avoid problems with footnotes in titles
\let\rmarkdownfootnote\footnote%
\def\footnote{\protect\rmarkdownfootnote}

%%% Change title format to be more compact
\usepackage{titling}

% Create subtitle command for use in maketitle
\newcommand{\subtitle}[1]{
  \posttitle{
    \begin{center}\large#1\end{center}
    }
}

\setlength{\droptitle}{-2em}

  \title{LAB5 - CP2}
    \pretitle{\vspace{\droptitle}\centering\huge}
  \posttitle{\par}
    \author{bruna barbosa}
    \preauthor{\centering\large\emph}
  \postauthor{\par}
    \date{}
    \predate{}\postdate{}
  

\begin{document}
\maketitle

\subsubsection{1. Imports}\label{imports}

\begin{Shaded}
\begin{Highlighting}[]
\KeywordTok{library}\NormalTok{(tidyverse)}
\KeywordTok{library}\NormalTok{(here)}
\KeywordTok{library}\NormalTok{(GGally)}
\KeywordTok{library}\NormalTok{(broom)}
\KeywordTok{library}\NormalTok{(plotly)}
\KeywordTok{library}\NormalTok{(dplyr)}
\KeywordTok{library}\NormalTok{(caret)}

\NormalTok{speed_dating =}\StringTok{ }\KeywordTok{read_csv}\NormalTok{(}\KeywordTok{here}\NormalTok{(}\StringTok{"data/speed-dating2.csv"}\NormalTok{))}

\NormalTok{speed_dating <-}\StringTok{ }\NormalTok{speed_dating %>%}
\StringTok{     }\KeywordTok{mutate}\NormalTok{(}\DataTypeTok{dec=}\KeywordTok{replace}\NormalTok{(dec, dec==}\StringTok{'yes'}\NormalTok{, }\DecValTok{1}\NormalTok{))}

\NormalTok{speed_dating <-}\StringTok{ }\NormalTok{speed_dating %>%}
\StringTok{     }\KeywordTok{mutate}\NormalTok{(}\DataTypeTok{dec=}\KeywordTok{replace}\NormalTok{(dec, dec==}\StringTok{'no'}\NormalTok{, }\DecValTok{0}\NormalTok{))}

\NormalTok{speed_dating$dec <-}\StringTok{ }
\StringTok{    }\KeywordTok{as.numeric}\NormalTok{(speed_dating$dec)}
\end{Highlighting}
\end{Shaded}

\subsubsection{2. Dentre os fatores que você acha que podem ter efeito
no match, quais fatores têm efeito significativo na chance de p1 decidir
se encontrar novamente com p2? E como é esse efeito
(positivo/negativo)?}\label{dentre-os-fatores-que-voce-acha-que-podem-ter-efeito-no-match-quais-fatores-tem-efeito-significativo-na-chance-de-p1-decidir-se-encontrar-novamente-com-p2-e-como-e-esse-efeito-positivonegativo}

Usaremos a função glm() para criar o modelo para a regressão logistica

Os fatores que mais influenciam na chance de p1 decidir se encontrar com
p2 sao aparencia fisica (attr) e quao divertido a pessoa é (fun).

A variavel dec possui dois valores: 0: p1 decidir nao se encontrar com
p2 novamente 1: p1 decidir se encontrar com p2 novamente

\begin{Shaded}
\begin{Highlighting}[]
\NormalTok{logit1 <-}\StringTok{ }\KeywordTok{glm}\NormalTok{(dec ~}\StringTok{ }\NormalTok{attr +}\StringTok{ }\NormalTok{fun, }\DataTypeTok{data =} \NormalTok{speed_dating, }\DataTypeTok{family =} \StringTok{"binomial"}\NormalTok{)}
\KeywordTok{summary}\NormalTok{(logit1)}
\end{Highlighting}
\end{Shaded}

\begin{verbatim}
## 
## Call:
## glm(formula = dec ~ attr + fun, family = "binomial", data = speed_dating)
## 
## Deviance Residuals: 
##     Min       1Q   Median       3Q      Max  
## -2.4000  -0.8703  -0.3787   0.9431   3.1209  
## 
## Coefficients:
##             Estimate Std. Error z value Pr(>|z|)    
## (Intercept) -5.71623    0.18799  -30.41   <2e-16 ***
## attr         0.55549    0.02523   22.02   <2e-16 ***
## fun          0.29836    0.02279   13.09   <2e-16 ***
## ---
## Signif. codes:  0 '***' 0.001 '**' 0.01 '*' 0.05 '.' 0.1 ' ' 1
## 
## (Dispersion parameter for binomial family taken to be 1)
## 
##     Null deviance: 6425.1  on 4714  degrees of freedom
## Residual deviance: 4969.2  on 4712  degrees of freedom
##   (203 observations deleted due to missingness)
## AIC: 4975.2
## 
## Number of Fisher Scoring iterations: 5
\end{verbatim}

\paragraph{2.1. Odds ratio}\label{odds-ratio}

Os coeficientes de attr e fun são positivos e significantes
estatisticamente. O log de odds ratio é mais dificil de interpretar, por
isso o expoente é utilizado.

\begin{Shaded}
\begin{Highlighting}[]
\KeywordTok{exp}\NormalTok{(}\KeywordTok{coef}\NormalTok{(logit1))}
\end{Highlighting}
\end{Shaded}

\begin{verbatim}
## (Intercept)        attr         fun 
## 0.003292094 1.742787436 1.347653468
\end{verbatim}

Neste caso vemos que se a beleza de uma pessoa é acrescida de 1 unidade,
sua probabilidade de ter um segundo encontro aumento em 1.74 ou 1.74\%.
E a diversao aumenta em 1 unidade, sua probabilidade de ter um segundo
encontro é 1.34 vezes maior. Como essa é uma regressao logistica com
multiplas variaveis, a mudança em uma variavel em 1 unidade considera
que as outras variaveis sao constantes. Por exemplo, quando o valor de
attr é acrescido em 1 unidade, o valor de fun é considerado como
constante.

\subsubsection{3. Que fatores nos dados têm mais efeito na chance de um
participante querer se encontrar novamente com
outro?}\label{que-fatores-nos-dados-tem-mais-efeito-na-chance-de-um-participante-querer-se-encontrar-novamente-com-outro}

\begin{Shaded}
\begin{Highlighting}[]
\NormalTok{logit2 <-}\StringTok{ }\KeywordTok{glm}\NormalTok{(dec ~}\StringTok{ }\NormalTok{attr +}\StringTok{ }\NormalTok{fun +}\StringTok{ }\NormalTok{intel +}\StringTok{ }\NormalTok{sinc +}\StringTok{ }\NormalTok{amb +}\StringTok{ }\NormalTok{shar +}\StringTok{ }\NormalTok{like, }\DataTypeTok{data =} \NormalTok{speed_dating, }\DataTypeTok{family =} \StringTok{"binomial"}\NormalTok{)}
\KeywordTok{summary}\NormalTok{(logit2)}
\end{Highlighting}
\end{Shaded}

\begin{verbatim}
## 
## Call:
## glm(formula = dec ~ attr + fun + intel + sinc + amb + shar + 
##     like, family = "binomial", data = speed_dating)
## 
## Deviance Residuals: 
##     Min       1Q   Median       3Q      Max  
## -2.6952  -0.7527  -0.2113   0.7800   3.7807  
## 
## Coefficients:
##             Estimate Std. Error z value Pr(>|z|)    
## (Intercept) -5.66262    0.26167 -21.640  < 2e-16 ***
## attr         0.40788    0.03047  13.388  < 2e-16 ***
## fun          0.13384    0.03299   4.057 4.97e-05 ***
## intel       -0.06289    0.04091  -1.537    0.124    
## sinc        -0.21058    0.03370  -6.249 4.12e-10 ***
## amb         -0.20513    0.03194  -6.423 1.34e-10 ***
## shar         0.19965    0.02585   7.722 1.14e-14 ***
## like         0.66400    0.04260  15.587  < 2e-16 ***
## ---
## Signif. codes:  0 '***' 0.001 '**' 0.01 '*' 0.05 '.' 0.1 ' ' 1
## 
## (Dispersion parameter for binomial family taken to be 1)
## 
##     Null deviance: 5618.7  on 4122  degrees of freedom
## Residual deviance: 3855.1  on 4115  degrees of freedom
##   (795 observations deleted due to missingness)
## AIC: 3871.1
## 
## Number of Fisher Scoring iterations: 5
\end{verbatim}

Dentre os atributos analisados, o quanto uma pessoa gostou de outra é o
fator mais importante para decidir se vai a um segundo encontro. Se o
like é acrescido em 1 unidade, a pessoas tem 0.66 vezes mais chances em
dizer sim para um segundo encontro. Em segundo lugar vem a beleza, neste
segundo modelo podemos ver que beleza apenas nao é o fator mais
importante para marcar um segundo encontro.


\end{document}
